\documentclass{scrreprt}

% Equations
\usepackage{amsmath}

% List customization
\usepackage[inline]{enumitem}

% They are all over the place
\usepackage{todonotes}

% Sub-figures
\usepackage{subcaption}

% Diagrams
\usepackage{tikz}
\usetikzlibrary{arrows,positioning}
\usepackage{pgfplots}
\usepgfplotslibrary{groupplots}
\usepackage{pgfgantt}

% inline arithmetic operations
\usepackage{fp}

% Appendices
\usepackage[header,title,titletoc]{appendix}
\renewcommand{\appendixname}{Appendix}

% License
\usepackage[
    type={CC},
    modifier={by-sa},
    version={3.0},
]{doclicense}

% Add bibliography to TOC
\usepackage[nottoc,numbib]{tocbibind}

% Sane datestamps
\usepackage[yyyymmdd]{datetime}
\renewcommand{\dateseparator}{--}

% Math symbols
\usepackage{amssymb}

% Haskell code highlighting
\usepackage{minted}
\newenvironment{code}{\VerbatimEnvironment\begin{minted}{haskell}}{\end{minted}}

% Include images
\usepackage{graphics}

\begin{document}

\title{Supporting Competent Authorities in the Implementation of the NIS
Directive for Safety-Critical Industries}
\subtitle{Modelling through attack-defense trees}
\author{Uma Zalakain \\ \href{mailto:2423394z@student.gla.ac.uk}{2423394z@student.gla.ac.uk}}

{\let\newpage\relax\maketitle}
\doclicenseThis
\newpage

\pagenumbering{roman}
\setcounter{tocdepth}{2}
\tableofcontents
\newpage
\pagenumbering{arabic}

\chapter{Introduction}

The EU directive on the security of Networks and Information Systems (NIS) has
been integrated into UK law. This directive aims to improve the reliability and
security of those network and information systems that are essential to the
economy, society and individual's welfare.
\footnote{\url{https://www.ncsc.gov.uk/guidance/introduction-nis-directive}}

The NIS directive does not require Operators of Essential Services (OES) to
conform to a particular set of fixed rules. Instead, it hands the responsibility
of oversight to the Competent Authorities (CAs). The NIS directive postulates 14
general principles
\footnote{\url{https://www.ncsc.gov.uk/guidance/table-view-principles-and-related-guidance}}
and CAs are strongly encouraged
\footnote{\url{https://www.gov.uk/government/publications/nis-regulations-guidance-for-competent-authorities}}
to use the Cyber Assessment Framework
\footnote{\url{https://www.ncsc.gov.uk/guidance/nis-directive-cyber-assessment-framework}}
to evaluate whether OES follow them. Depending on the sector, a CA may issue
more specific guidance. One such example is the guidance for Cyber Security for
Industrial Automation and Control Systems
\footnote{\url{http://www.hse.gov.uk/foi/internalops/og/og-0086.pdf}} issued by
the Health and Safety Executive.

\section{Contributions}

In this report I present an embedded domain specific language
\footnote{\url{https://en.wikipedia.org/wiki/Domain-specific_language}} for
modelling attack-defense trees \cite{Kordy} and performing quantitative analysis
and decision making on them. I then demonstrate how it helps a CA like Ofcom
(designated regulator for the digital infrastructure subsector) guide and assess
that an OES like Nominet \footnote{\url{https://nominet.uk}} (provider of the
top-level domain \texttt{.uk}) complies with the NIS directive by following the
guidance laid out by the Cyber Assessment Framework.

\todo{Which of the 12 points are we covering}

\chapter{An eDSL for attack defense trees}

\section{Attack defense trees}

Fault trees were first introduced as a graphical and logical tool for modelling
the way in which basic events interact to lead to an undesired event
\cite[IV.1]{Vesely1981}. They are a tree of nodes and leafs, where each node is
a logical gate and each leaf a basic event. This structure makes fault trees
attractive for quantitative evaluation. It is important to note that all basic
events in a fault tree \textbf{must} be unique and independent. Every directed
acyclic graph representing a fault model can be rewritten as a fault tree.
\todo{citation needed}

Attack trees take the idea of fault trees and adapt it to model attack scenarios
in security-critical systems \cite{Schneier1999} \cite{Brooke2003}
\cite{NaiFovino2009}: an attacker's objective (at the root of the tree) is
repeatedly refined into subgoals connected by logical gates -- it is unable to
represent any defensive measure.

Attack defense trees (ADTs) extend attack trees by allowing nodes to represent
defensive measures \cite{Kordy}. Every node is thus an action of either the
attacker or the defender: interactive scenarios can be represented. As with
fault trees, refinements can either be disjunctive or conjunctive. To allow a
quantitative evaluation of ADTs, nodes have to be independent of each other.
Alternative proposals relying on DAGs exist, but are more complex
\cite{Kordy2013a}.

\includegraphics[width=\textwidth]{bankaccount}

% TODO: figure with an example ADT

% TODO: relate to NIS: CAF points
%   IMPORTANT: How attack defense trees help

\section{Requirements}

% Simple modelling

% Generalised attributes
\cite{Kordy2012}
%   Generalised algebra on attributes
%   Quantitative analysis

% Tree to graph

\section{Implementation}

\chapter{Evaluation}

% Showcase the tool with an example
% Underline how it helps implementing NIS
% We need to conduct some kind of experiment

% https://www.ofcom.org.uk/phones-telecoms-and-internet/information-for-industry/guidance-network-information-systems-regulations
% https://www.ncsc.gov.uk/information/active-cyber-defence-one-year
% https://www.theregister.co.uk/2017/08/17/uks_internet_operator_nominet_goes_dark/

\chapter{Results}

\chapter{Conclusions}

\bibliographystyle{apalike}
\bibliography{bibliography}

\newpage
\begin{appendices}
% Only display section titles in the TOC
\addtocontents{toc}{\protect\setcounter{tocdepth}{1}} 

\chapter{Source code}
% Where to obtain it
% How to install it
% A listing

\end{appendices}

\end{document}
