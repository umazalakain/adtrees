\documentclass{scrreprt}

% Equations
\usepackage{amsmath}

% List customization
\usepackage[inline]{enumitem}

% They are all over the place
\usepackage{todonotes}

% Sub-figures
\usepackage{subcaption}

% Diagrams
\usepackage{tikz}
\usetikzlibrary{arrows,positioning}
\usepackage{pgfplots}
\usepgfplotslibrary{groupplots}
\usepackage{pgfgantt}

% inline arithmetic operations
\usepackage{fp}

% Appendices
\usepackage[header,title,titletoc]{appendix}
\renewcommand{\appendixname}{Appendix}

% License
\usepackage[
    type={CC},
    modifier={by-sa},
    version={3.0},
]{doclicense}

% Add bibliography to TOC
\usepackage[nottoc,numbib]{tocbibind}

% Sane datestamps
\usepackage[yyyymmdd]{datetime}
\renewcommand{\dateseparator}{--}

% Math symbols
\usepackage{amssymb}

\begin{document}

\title{Supporting Competent Authorities in the Implementation of the NIS
Directive for Safety-Critical Industries}
\subtitle{Modelling through attack-defense trees}
\author{Uma Zalakain \\ \href{mailto:2423394z@student.gla.ac.uk}{2423394z@student.gla.ac.uk}}

{\let\newpage\relax\maketitle}
\doclicenseThis
\newpage

\pagenumbering{roman}
\setcounter{tocdepth}{2}
\tableofcontents
\newpage
\pagenumbering{arabic}

\chapter{Introduction}

The EU directive on the security of Networks and Information Systems (NIS) has
been integrated into UK law. This directive aims to improve the reliability and
security of those network and information systems that are essential to the
economy, society and individual's welfare.
\footnote{\url{https://www.ncsc.gov.uk/guidance/introduction-nis-directive}}

The NIS directive does not require Operators of Essential Services (OES) to
conform to a particular set of fixed rules. Instead, it hands the responsibility
of oversight to the Competent Authorities (CAs). The NIS postulates 14 general
principles
\footnote{\url{https://www.ncsc.gov.uk/guidance/table-view-principles-and-related-guidance}}
and CAs are strongly encouraged
\footnote{\url{https://www.gov.uk/government/publications/nis-regulations-guidance-for-competent-authorities}}
to use the Cyber Assessment Framework
\footnote{\url{https://www.ncsc.gov.uk/guidance/nis-directive-cyber-assessment-framework}}
to evaluate the OES' compliance to them. Depending on the sector, a CA may issue
more specific guidance. One such example is the guidance for Cyber Security for
Industrial Automation and Control Systems
\footnote{\url{http://www.hse.gov.uk/foi/internalops/og/og-0086.pdf}} issued by
the Health and Safety Executive.

\section{Contributions}

In this report I present an embedded domain specific language
\footnote{\url{https://en.wikipedia.org/wiki/Domain-specific_language}} for
modelling attack-defense trees \cite{Kordy} and performing quantitative analysis
and decision making on them. I then demonstrate how it helps a CA like Ofcom
(designated regulator for the digital infrastructure subsector) guide and assess
an OES like Nominet \footnote{\url{https://nominet.uk}} (provider of the
top-level domain \texttt{.uk}) comply with the NIS directive by following the
guidance layed out by the Cyber Assessment Framework.

\chapter{Tool}

% How attack defense trees help
% Fault trees are for safety (must be independent)
% Security: attack defense trees
% Every node is an action of the attacker or the defender
% Nodes must be independent
% Diamond shapes can be reformulated
% DAGS, necesarry for dependencies and probabilistic viewpoints

\cite{Kordy2013a}

\chapter{Evaluation}

% Showcase the tool with an example
% Underline how it helps implementing NIS
% We need to conduct some kind of experiment

% https://www.ofcom.org.uk/phones-telecoms-and-internet/information-for-industry/guidance-network-information-systems-regulations
% https://www.ncsc.gov.uk/information/active-cyber-defence-one-year
% https://www.theregister.co.uk/2017/08/17/uks_internet_operator_nominet_goes_dark/

\chapter{Results}

\chapter{Conclusions}

\bibliographystyle{ieeetr}
\bibliography{bibliography}

\newpage
\begin{appendices}
% Only display section titles in the TOC
\addtocontents{toc}{\protect\setcounter{tocdepth}{1}} 

\section{Source code}
% Where to obtain it
% How to install it
% A listing

\end{appendices}

\end{document}
